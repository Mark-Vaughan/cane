\documentclass[10pt,a4paper]{article}
\usepackage[utf8]{inputenc}
\usepackage{amsmath}
\usepackage{amsfonts}
\usepackage{amssymb}
\usepackage{import}
\usepackage{graphicx}
\usepackage{color}
\usepackage[]{algorithm2e}

\setlength{\parindent}{0cm}

% Define own macros
\newcommand{\myfrac}[2]{%
    \ifinner#1/#2%
    \else\frac{#1}{#2}%
    \fi%
}

\title{Signorini frictionless contact problem}

\author{Dr.-Ing. Andreas Apostolatos}

\date{\today}

\begin{document}

\maketitle

\section{Theory}\label{sec:variational_formulation}

\subsection{Linear elasticity}\label{subsec:linear_elasticity}

Given is a deformable body which is geometrically described by $\Omega \subset \mathbb{R}^d$ where $d = 2,3$ stands for the number of the problem's spatial dimensions. Its deformation due to the applied boundary conditions can be uniquely defined by a displacement field $\mathbf{u} = u_i \, \mathbf{e}_i$, $\mathbf{e}_i$ being the Cartesian basis, which maps each each material point of the reference configuration $\mathbf{X} \in \Omega$ to a material point in the current configuration $\mathbf{x} \in \Omega_t$, that is, $\mathbf{x} = \mathbf{X} + \mathbf{u}$.\\

The strain is described by means of the \textit{Green-Lagrange} (GL) strain second order tensor $\boldsymbol{\mathcal{E}} \in \mathfrak{S}^2$ given by,

\begin{equation}
	\boldsymbol{\mathcal{E}} = \frac{1}{2} \left( \boldsymbol{\nabla} \mathbf{u} + \left( \boldsymbol{\nabla} \mathbf{u} \right)^{\text{t}} +  \boldsymbol{\nabla} \mathbf{u} \cdot \left( \boldsymbol{\nabla} \mathbf{u} \right)^{\text{t}} \right) \;, \label{eq:GL_strain}
\end{equation}

where the underlying symbols are understood as follows,

\begin{subequations}
	\begin{alignat}{1}
		\boldsymbol{\nabla} \mathbf{u} &= \sum_{i = 1}^d \sum_{j = 1}^d \frac{\partial u_i}{\partial X_j} \, \mathbf{e}_i \otimes \mathbf{e}_j \;, \label{eq:nabla_u} \\
		\left( \boldsymbol{\nabla} \mathbf{u} \right)^{\text{t}} &= \sum_{k = 1}^d \sum_{l = 1}^d \frac{\partial u_l}{\partial X_k} \, \mathbf{e}_k \otimes \mathbf{e}_l \;, \label{eq:nabla_u_t} \\
		\boldsymbol{\nabla} \mathbf{u} \cdot \left( \boldsymbol{\nabla} \mathbf{u} \right)^{\text{t}} &= \sum_{i = 1}^d \sum_{j = 1}^d \sum_{k = 1}^d \frac{\partial u_i}{\partial X_j} \frac{\partial u_j}{\partial X_k} \, \mathbf{e}_i \otimes \mathbf{e}_k \;. \label{eq:nabla_u_times_u_t}
	\end{alignat}
\end{subequations}

Assumed is also a Saint-Venant material model which is governed by the material fourth order tensor $\boldsymbol{\mathcal{C}} \in \mathfrak{S}^2$

\begin{equation}
	\boldsymbol{\mathcal{C}} = \frac{E}{2 (1 + \nu)} \left( \delta_{ik} \delta_{jl} + \delta_{il} \delta_{jk} + \frac{2 \nu}{1 - 2 \nu} \delta_{ij} \delta_{kl} \right) \, \mathbf{e}_i \otimes \mathbf{e}_j \otimes \mathbf{e}_k \otimes \mathbf{e}_l \;, \label{eq:material_tensor}
\end{equation}

where $E$ and $\nu$ stand for the Young's modulus and Poisson ratio of the material, respectively. The stress state of the problem is described by means of the 2nd \textit{Piola-Kirchhoff} (PK2) stress second order ternsor $\boldsymbol{\mathcal{S}} \in \mathfrak{S}^2$ which is defined as and given by the linear elastic isotropic law as,

\begin{equation}
	\boldsymbol{\mathcal{S}} = \boldsymbol{\mathcal{C}}  : \boldsymbol{\mathcal{E}} \:.
\end{equation}

\begin{figure}[!t]
	\centering
	\footnotesize
    \def\svgwidth{0.5\textwidth}\import{./figures/}{problem_placement.pdf_tex}
	\caption{Theory: Signorini frictionless contact problem with boundary conditions.}
	\label{im:signorini_contact_problem}
\end{figure}

Given the Dirichlet boundary conditions along $\Gamma_{\text{d}}$ where the displacement is prescribed to a value $\bar{\mathbf{u}}$ which is assumed to be zero without loss of generality, the Neumann boundary conditions along $\Gamma_{\text{n}}$ where external traction $\bar{\mathbf{t}}$ is applied and applied body forces $\mathbf{b}$, see Fig.~\ref{im:signorini_contact_problem}, the strong form of the elasticity problem writes,

\begin{subequations}
	\begin{alignat}{2}
		\boldsymbol{\nabla} \cdot \boldsymbol{\mathcal{S}} + \mathbf{b} &= \mathbf{0} \quad &&\text{in } \Omega \;, \label{eq:strong_equilibrium} \\
		\mathbf{u} &= \bar{\mathbf{u}} &&\text{on } \Gamma_{\text{d}} \;, \label{eq:prescribed_displacement} \\
		\boldsymbol{\mathcal{S}} \cdot \mathbf{n} &= \bar{\mathbf{t}} &&\text{on } \Gamma_{\text{n}} \;,
	\end{alignat}
	\label{eq:strong_formulation}
\end{subequations}

where $\mathbf{n}$ stands for the unit outward normal vector to Neumann boundary $\Gamma_{\text{n}}$.

\subsection{Contact conditions}\label{subsec:contact_conditions}

Assumed is also that there is a boundary $\Gamma_{\text{c}} = \Gamma_{\text{c}} (\mathbf{u})$ along which body $\Omega$ is expected to come into contact with a rigid body $\mathfrak{F}$ along a portion of its surface $\mathbb{S} \subset \partial \mathfrak{F}$. Assumed is that for each material particle $\mathbf{X} \in \Gamma_{\text{c}}$ there exists a unique material particle $\mathbf{Z} \in \mathbb{S}$ which is closest to $\mathbf{X}$ in Euclidean sense,

\begin{equation}
	\mathbf{Z} (\mathbf{X}) = \arg \min_{\mathbf{Y} \in \Gamma_{\text{c}}} \left\| \mathbf{X} - \mathbf{Y} \right\|_2 \;. \label{eq:euclidean_distance}
\end{equation}

The so-called gap function is then defined along $\Gamma_{\text{c}}$, namely $\mathbf{g}:\Gamma_{\text{c}} \rightarrow \mathbb{R}$ and measures the normal distance between the deformable and the rigid body $\Omega$ along $\Gamma_{\text{c}}$ and $\mathbb{S}$,

\begin{equation}
	g = u_n - g_0 \quad \text{for all }\mathbf{X} \in \Gamma_{\text{c}} \:, \label{eq:gap_function}
\end{equation}

where $u_n = \mathbf{u} \cdot\mathbf{n}$ is the normal component of the displacement field given the outward normal $\mathbf{n}$ to $\Gamma_{\text{c}}$ at $\mathbf{X} \in \Gamma_{\text{c}}$ and $g_0(\mathbf{X})$ is the initial gap given by,

\begin{equation}
	g_0(\mathbf{X}) = - (\mathbf{X} - \mathbf{Z}(\mathbf{X})) \cdot \mathbf{n}(\mathbf{X}) \quad \text{for all }\mathbf{X} \in \Gamma_{\text{c}} \;. \label{eq:initial_gap}
\end{equation}

Herein it is assumed that neither the closest point $\mathbf{Z}$ nor the outward normal $\mathbf{z}$ depend on the displacement field $\mathbf{u}$ but solely on the point $\mathbf{X} \in \Gamma_{\text{c}}$ and that $\mathbf{k} (\mathbf{X}) = - \mathbf{n} (\mathbf{X})$, where $\mathbf{k} (\mathbf{X})$ stands for the outward normal to $\mathbb{S}$ at $\mathbf{Z} \in \mathbb{S}$. These assumptions are valid for sufficiently small gaps. Moreover, the following conditions are assumed,

\begin{itemize}

	\item[i.] No material particle can penetrate the body $\mathfrak{F}$, hence $g(\mathbf{X}) \le 0$ for all $\mathbf{X} \in \Gamma_{\text{c}}$,
	
	\item[ii.] Surface $\mathbb{S}$ is sufficiently lubricated such that no shear tractions develop, that is, $t_t = \mathbf{n} \cdot \boldsymbol{\mathcal{S}} \cdot \mathbf{t} = 0$ where $\mathbf{t}$ stands for the normal tangent vector to $\mathbb{S}$,
	
	\item[iii.] The normal tractions along $\Gamma_{\text{c}}$ are compressive, that is, $t_n = \mathbf{n} \cdot \boldsymbol{\mathcal{S}} \cdot \mathbf{n} \ge 0$,
	
	\item[iv.] If there is contact along $\Gamma_{\text{c}}$ then $g = 0$ and $t_n \ge 0$ whereas is there is no contact along $\Gamma_{\text{c}}$ then $g \le 0$ and $t_n = 0$

\end{itemize}

In this way, the so-called complementarity conditions can be stated as follows,

\begin{subequations}
	\begin{alignat}{1}
		g &\le 0 \;, \label{eq:complementarity_cnd_gap} \\
		t_n &\ge 0 \;,  \label{eq:complementarity_cnd_traction} \\
		t_n\,g &= 0 \:. \label{eq:complementarity_cnd_mixed} 
	\end{alignat}
	\label{eq:complementarity_cnds}
\end{subequations}

Condition in Eq.~\eqref{eq:complementarity_cnd_gap} stands for the non-penetration condition whereas condition in Eq.~\eqref{eq:complementarity_cnd_traction} states that the stresses along the contact boundary need to be compressive. Eq.~\eqref{eq:complementarity_cnd_mixed} is a direct consequence of condition iv.

\subsection{Variational formulation}\label{subsec:variational_formulation}

The weak formulation of the problem in Eq.~\eqref{eq:strong_formulation} subject to the complementarity conditions in Eq.~\eqref{eq:complementarity_cnds} can be written by means of the Langrange Multipliers method: Find $\mathbf{u} \in \boldsymbol{\mathcal{H}}^1 (\Omega)$ with $\mathbf{u} = \mathbf{g}$ on $\Gamma_{\text{d}}$ and $\lambda \in \mathcal{L}^2(\Gamma_{\text{c}})$ with $\lambda \ge 0$ such that,

\begin{subequations}
	\begin{alignat}{1}
		\int_{\Omega} \delta \boldsymbol{\mathcal{E}} : \boldsymbol{\mathcal{S}} \: \text{d} \Omega + \int_{\Gamma_{\text{c}}} \delta u_n \, \lambda \: \text{d} \Gamma &= \int_{\Omega} \delta \mathbf{u} \cdot \mathbf{b} \: \text{d} \Omega + \int_{\Gamma_{\text{n}}} \bar{\mathbf{t}} \cdot \mathbf{u} \: \text{d} \Gamma \label{eq:variational_formulation_delta_u} \\
		\int_{\Gamma_{\text{c}}} \delta \lambda \, u_n \: \text{d} \Gamma &= \int_{\Gamma_{\text{c}}} \delta \lambda \, g_0 \: \text{d} \Gamma \;, \label{eq:variational_formulation_delta_lambda}
	\end{alignat}
	\label{eq:variational_problem_signorini}
\end{subequations}

for all $\delta \mathbf{u} \in \boldsymbol{\mathcal{H}}^1 (\Omega)$ and for all $\delta \lambda \in \mathcal{L}^2(\Gamma_{\text{c}})$ with $\delta \lambda \ge 0$. In fact, $\lambda$ in this case represents the normal traction along the contact boundary $\Gamma_{\text{c}}$ and therefore it must be positive, see Eq.~\eqref{eq:complementarity_cnd_traction}. Eq.~\eqref{eq:variational_formulation_delta_u} is nothing else but the internal virtual work in $\Omega$, whereas Eq.~\eqref{eq:variational_formulation_delta_lambda} accounts for the variation of complementarity condition in Eq.~\eqref{eq:complementarity_cnd_mixed} and renders the problem symmetric as it is also demonstrated in its discrete form.\\

Important is to note that the variational problem in Eq.~\eqref{eq:variational_problem_signorini} is valid only when $\Gamma_{\text{c}} \neq \emptyset$. The latter means that there has to be a non-empty set $\Gamma_{\text{c}}$ along which the complementarity conditions in Eq.~\eqref{eq:complementarity_cnds} are valid. Otherwise, in case that $\Gamma_{\text{c}} = \emptyset$ variational problem in Eq.~\eqref{eq:variational_problem_signorini} reduces to: Find $\mathbf{u} \in \boldsymbol{\mathcal{H}}^1 (\Omega)$ with $\mathbf{u} = \mathbf{g}$ on $\Gamma_{\text{d}}$ such that,

\begin{equation}
	\int_{\Omega} \delta \boldsymbol{\mathcal{E}} : \boldsymbol{\mathcal{S}} \: \text{d} \Omega = \int_{\Omega} \delta \mathbf{u} \cdot \mathbf{b} \: \text{d} \Omega + \int_{\Gamma_{\text{n}}} \bar{\mathbf{t}} \cdot \mathbf{u} \: \text{d} \Gamma \quad \text{for all } \delta \mathbf{u} \in \boldsymbol{\mathcal{H}}^1 (\Omega) \;. \label{eq:variational_formulation_structural_mechanics} \\
\end{equation} 

\section{Discretization}

Assumed is that the displacement field is discretized with the standard linear \textit{Finite Element Method} (FEM) on triangles or quadrilaterals. Employing the Buvnon-Galerkin FEM then the unknown displacement field $\mathbf{u}$ and its variation $\delta \mathbf{u}$ are discretized using the same basis functions $\boldsymbol{\varphi}_i$, that is,

\begin{subequations}
	\begin{alignat}{1}
		\mathbf{u}_h &= \sum_{i = 1}^{n} \boldsymbol{\varphi}_i \hat{u}_i \;, \label{eq:discretization_u} \\
		\delta \mathbf{u}_h &= \sum_{i = 1}^n \boldsymbol{\varphi}_i \delta\hat{u}_i \;, \label{eq:discretization_delta_u}
	\end{alignat}
\end{subequations}

where the linear/bilinear basis functions $\boldsymbol{\varphi}_i$ are constructed as a dual product of the linear/bilinear shape functions $N_i$ at the element's parametric space and the Cartesian basis $\mathbf{e}_i$. Accordingly, $\hat{u}_i$ and $\delta\hat{u}_i$ stand for the \textit{Degrees of Freedom} (DOFs) of the unknown and the test fields, respectively, and $n \in \mathbb{N}$ is the number of nodes in the mesh. Subscript $h$ indicates then the smallest element length size within the employed mesh.\\

Within the node-based contact method, the shape functions for the discretization of the Lagrange Multipliers fields $\lambda$ and $\delta \lambda$ are chosen the diract delta distributions supported on the finite element nodes $\mathbf{X}_i$, that is,

\begin{subequations}
	\begin{alignat}{1}
		\lambda_h &= \sum_{i = 1}^{n_{\text{c}}} \delta(\mathbf{X}_i) \hat{\lambda}_i \;, \label{eq:discretization_lambda} \\
		\delta \lambda_h &= \sum_{i = 1}^{n_{\text{c}}} \delta(\mathbf{X}_i) \delta\hat{\lambda}_i \;, \label{eq:discretization_lambda_u}
	\end{alignat}
\end{subequations}

where as before $\hat{\lambda}_i$ and $\delta\hat{\lambda}_i$ stand for the Lagrange Multipliers DOFs of the unknown and test fields, respectively. Note that altough there are two DOFs per node for the displacement field, there is only one DOF per node for the Lagrange Multipliers field. This is because the displacement field is a vector field in the $\mathbb{R}^2$ space in two-dimensional elasticity whereas the Lagrange Multipliers field is a scalar field due to the complementarity constraint in Eq.~\eqref{eq:complementarity_cnd_mixed} which is a scalar-valued constraint. In case of a vector-valued constraint, then the Lagrange Multipliers field would also be a vector field.\\

\begin{figure}[!t]
	\centering
	\footnotesize
    \def\svgwidth{0.7\textwidth}\import{./figures/}{finite_element_mesh.pdf_tex}
	\caption{Discretization: Finite element mesh and degrees of freedom.}
	\label{im:finite_element_mesh}
\end{figure}

Substituting the latter expressions in Eq.~\eqref{eq:variational_problem_signorini} the following discrete system of equations is obtained in terms of a Newton-Raphson formulation,

\begin{equation}
	\left[ \begin{array}{cc}
		\mathbf{K} ( \hat{\mathbf{u}}_{\hat{i}} , \hat{\boldsymbol{\lambda}}_{\hat{i}} ) & \mathbf{C}^{\text{t}} \\
		\mathbf{C} & \mathbf{0}
\end{array}	 \right] \, \Delta_{\hat{i}} \, \left[ \begin{array}{c}
	\hat{\mathbf{u}} \\
	\hat{\boldsymbol{\lambda}}
\end{array} \right] = - \left[ \begin{array}{c}
	\mathbf{R} ( \hat{\mathbf{u}}_{\hat{i}} , \hat{\boldsymbol{\lambda}}_{\hat{i}} ) \\
	\boldsymbol{\mathcal{R}} ( \hat{\mathbf{u}}_{\hat{i}} )
\end{array} \right] \;, \label{eq:system_discrete_form_lagrange_multipliers}
\end{equation}

over the so-called active set of Lagrange Multipliers DOFs. The $i$-th component of the residual vector $\mathbf{R}$ derived from Eq.~\eqref{eq:variational_formulation_delta_u} is defined as follows,

\begin{equation}
	R_i ( \hat{\mathbf{u}}_{\hat{i}} , \hat{\boldsymbol{\lambda}}_{\hat{i}} ) = \int_{\Omega} \frac{\partial \boldsymbol{\mathcal{E}}}{\partial \hat{u}_i} : \boldsymbol{\mathcal{S}} \: \text{d} \Omega + \sum_{j = 1}^{n_{\text{c}}} \bar{n} (\mathbf{X}_j) \, \hat{\lambda}_j - \int_{\Omega} \boldsymbol{\varphi}_i \cdot \mathbf{b} \: \text{d} \Omega - \int_{\Omega} \boldsymbol{\varphi}_i \cdot \bar{\mathbf{t}} \: \text{d} \Gamma \;. \label{eq:residual_form_domain}
\end{equation}

The second term in Eq.~\eqref{eq:residual_form_domain} is derived by inserting Eq.~\eqref{eq:discretization_lambda} in Eq.~\eqref{eq:variational_formulation_delta_u} and by using the well-known identity of the Dirac delta distribution, that is,

\begin{align}
	\int_{\Gamma_{\text{c}}} \delta u_n \, \lambda \: \text{d} \Gamma \approx \int_{\Gamma_{\text{c}}} \delta \mathbf{u}_h \cdot \mathbf{n} \, \lambda_h \: \text{d} \Gamma \nonumber = \sum_{i = 1}^{n} \sum_{j = 1}^{n_{\text{c}}} \delta \hat{u}_i \int_{\Gamma_{\text{c}}} \boldsymbol{\varphi}_i \cdot \mathbf{n} \; \delta (\mathbf{X}_j) \hat{\lambda}_j  \, \text{d} \Gamma = \nonumber \\
	\sum_{i = 1}^n \sum_{j = 1}^{n_{\text{c}}} \delta \hat{u}_i \, \boldsymbol{\varphi}_i (\mathbf{X}_j) \cdot \mathbf{n} \, \hat{\lambda}_j = \sum_{j = 1}^{n_{\text{c}}} \delta \hat{u}_j \, \mathbf{e}_{\text{mod} \left( \frac{j - 1}{2} \right) + 1} \cdot \mathbf{n}  \, \hat{\lambda}_j = \sum_{j = 1}^{n_{\text{c}}} \delta \hat{u}_j \, \bar{n} (\mathbf{X}_j) \, \hat{\lambda}_j \nonumber \;,
\end{align}

where $\bar{n} (\mathbf{X}_j) = n_1 (\mathbf{X}_j) + n_2 (\mathbf{X}_j)$ is nothing else but the sum of the components of the outward normal vector $\mathbf{n}$ to the boundary $\Gamma_{\text{c}}$ at node $\mathbf{X}_j \in \Gamma_{\text{c}}$.\\

It is worth noting that within linear FEM, the boundary $\Gamma_{\text{c}}$ is only piecewise continuous which means that the the outward normal vector $\mathbf{n}$ to $\Gamma_{\text{c}}$ is not uniquely defined at the finite element nodes $\mathbf{X}_i$. Therefore, $\mathbf{n}(\mathbf{X}_i)$ is constructed by averaging the outward normal vectors from the neighbouring elements along $\Gamma_{\text{c}}$.\\

The $i$-th component of the residual vector $\boldsymbol{\mathcal{R}}$ derived from Eq.~\eqref{eq:variational_formulation_delta_lambda} is defined as follows,

\begin{equation}
	\mathcal{R}_i ( \hat{u}_i ) = \bar{n} (\mathbf{X}_i) \, \hat{u}_i - g_0(\mathbf{X}_i) \;.
\end{equation}

Then, the tangent stiffness matrix $\mathbf{K}$ has components,

\begin{equation}
	K_{ij} ( \hat{\mathbf{u}}_{\hat{i}} ) = \frac{\partial R_i}{\partial \hat{u}_j} = \int_{\Omega} \frac{\partial \boldsymbol{\mathcal{E}}}{\partial \hat{u}_i} : \frac{\partial \boldsymbol{\mathcal{S}}}{\partial \hat{u}_j} + \frac{\partial^2 \boldsymbol{\mathcal{E}}}{\partial \hat{u}_i \partial \hat{u}_j} : \boldsymbol{\mathcal{S}} \: \text{d} \Omega \:. \label{eq:tangent_stiffness_matrix}
\end{equation}

The Lagrange Multipliers matrix $\mathbf{C}$ is then a diagonal matrix due to the choice of the discretization for the Lagrange Multipliers field and has entries,

\begin{equation}
	C_{ii} = \bar{n} (\mathbf{X}_i) \:,	\label{eq:lagrange_multipliers_matrix}
\end{equation}

at each active node $\mathbf{X}_i \in \Gamma_{\text{c}}$. Vectors $\hat{\mathbf{u}}$ and $\hat{\boldsymbol{\lambda}}$ stand for the collection of all the displacement and Lagrange Multipliers DOFs as follows,

\begin{subequations}
	\begin{alignat}{1}
		\hat{\mathbf{u}} &= \left[ \begin{array}{ccc} \hat{u}_1 & \cdots & \hat{u}_n \end{array} \right] \;, \label{eq:vector_of_displacement_dofs} \\
		\hat{\mathbf{u}} &= \left[ \begin{array}{ccc} \hat{\lambda}_1 & \cdots & \hat{\lambda}_{n_{\text{c}}} \end{array} \right] \;. \label{eq:vector_of_lagrange_multipliers_dofs}
	\end{alignat}
	\label{eq:vectors_of_dofs}
\end{subequations}

The index $\hat{i}$ on $\hat{\mathbf{u}}$ and $\hat{\boldsymbol{\lambda}}$ indicates the Newton-Raphson iteration and $\Delta_{\hat{i}} (\bullet) = (\bullet)_{\hat{i} + 1} - (\bullet)_{\hat{i}}$.

\section{Realization}

As aforementioned, the actual contact surface $\Gamma_{\text{c}}$ is not known in advanced, since it depends on the solution $\mathbf{u}$. Given the variational inequality nature of the problem, an iterative approach is employed. This procedure is illustrated in Algorithm~\ref{alg:contact_algorithm}.

\begin{algorithm}[H]

	compute the initial gap function in Eq.~\eqref{eq:initial_gap} \; \vspace*{5pt}

	\While{conditions in~\eqref{eq:complementarity_cnd_gap}-~\eqref{eq:complementarity_cnd_traction} are not satisfied}{
	\vspace*{5pt}
	
	solve the elasticity problem in Eq.~\eqref{eq:variational_formulation_structural_mechanics} \;	\vspace*{5pt}
	
	find the nodes penetrating the rigid body $\mathfrak{F}$ \; \vspace*{5pt}
	
	update the vector of Lagrange Multipliers in Eq.~\eqref{eq:discretization_lambda} with the new constraints \; \vspace*{5pt} 
	
	compute the normal traction at the contact nodes using the Lagrange Multipliers solution \;	\vspace*{5pt}	
	
	compute the new Lagrange Multipliers matrix in Eq.~\eqref{eq:lagrange_multipliers_matrix} \vspace*{5pt} \;
	
	assemble the new constrained by Lagrange Multipliers system in Eq.~\eqref{eq:system_discrete_form_lagrange_multipliers} \; \vspace*{5pt}
	
	
	}

	\caption{Contact algorithm}
	
	\label{alg:contact_algorithm}
\end{algorithm}

% \bibliographystyle{elsarticle-num-names}
\bibliographystyle{alpha}
\bibliography{literature}
\nocite{*}

\end{document}